\documentclass{article}

\usepackage{homework}

\course{Encryption}
\assignment{Homework 3}

\begin{document}
	\homeworktitle
	
	\begin{enumerate}[label=\textbf{Task \arabic*:}]
		\item \textbf{RSA cryptosystem}\\
		The paper claims that with having $\phi(n)$ it is trivial to break RSA. But it does not tell anything about how one might gain $\phi(n)$, which is not easy for large numbers, that are generally used with RSA. Finding the Totient function ($\phi(n)$) generally requires to calculate all the primes that are smaller than $n$, which is a time consuming task.\\
		My advice based on this paper alone would be to keep using the already existing encryption system.
		\item \textbf{Security definitions}\\
		%What is the IND-OT-CPA? The adversary knows the public key(pk, n), sends a pair of messages m1 and m2, and gets one ciphertext back. Then he has to get which message was encrypted.
		
		\item \textbf{ElGamal encryption}
		\item \textbf{Attack on RSA encryption}
		\item \textbf{Choosing cryptographic primitives}
	\end{enumerate}
	
\end{document}