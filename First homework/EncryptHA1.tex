\documentclass{article}

\usepackage{homework}

\course{Encryption}
\assignment{Homework 1}

\begin{document}
\homeworktitle

\begin{enumerate}[label={}]
	\item \textbf{Task 1:} Affine cipher in $\Z_{97}$.
	\begin{enumerate}
		\item Encryption and Decryption functions with key $k=(a,b)$ (the message is $m$ and the resulting ciphertext is $c$):
		\begin{align*}
			E(k, m) &= a \cdot m + b \bmod 97\\
			D(k, c) &= a^{-1} \cdot (c - b) \bmod 97
		\end{align*}
		\item The number of possible meaningful and different keys is \[ \lvert a \rvert \cdot \lvert b \rvert \]
		Where $\lvert a \rvert$ is the possibilities for $a$ and $\lvert b \rvert$ is the different possibilities for $b$. $b$ can have 97 different values, and $a$ can have as many as there are co-primes below 97 with 97. This is given by Euler's Totient function: $\phi(97) = 96$ (this is because 97 is prime).\\
		In total the number of different keys is: 
		\[96 \cdot 97 = 9312\]
		\item We have an intercepted ciphertext-plaintext pair: (DOG)=(28 83 43), and a ciphertext with the same key: (78 23 33).\\
		To figure out the key we can use two letters from the known ciphertext-plaintext pair with either the Encryption or the Decription functions: (I choose the decryption with letters (DO) = (28 83))
		\begin{align*}
			D(k, 28) &= a^{-1} \cdot (28 - b) \bmod 97 = (D)3\\
			D(k, 83) &= a^{-1} \cdot (83 - b) \bmod 97 = (O)14
		\end{align*}
		Thus we get:
		\begin{align*}
				a^{-1} \cdot (28 - b) \bmod 97 &= 3\\
				a^{-1} \cdot (83 - b) \bmod 97 &= 14
		\end{align*}
		Rewriting the equations we get:
		\begin{align*}
			b &= 28 - 3a \bmod 97\\
			b &= 83 - 14a \bmod 97
		\end{align*}
		Making the two equal each other:
		\[28 - 3a = 83 - 14a \bmod 97\]
		With rearranging we get:
		\[55 = 11a \bmod 97\]
		We need the multiplicative inverse of 11 in $\bmod 97$, which is 53. (Using the Extended Euclidean Algorithm this is fairly straightforward.) With multiplication we get:
		\[a = 55 \cdot 53 \bmod 97 = 2915 \bmod 97 = 5\]
		Substituting back to one previous equation we get:
		\[b = 28 - 3 \cdot 5 \bmod 97 = 28 -15 \bmod 97 = 13\]
		Meaning the key is: $k = (5, 13)$\\
		For decrypting the second message we firstly need the multiplicative inverse of $5 \pmod 97$ which is 39. Writing this into the decryption function we get:
		\begin{align*}
			D(k, 78) &= 39 \cdot (78 - 13) \bmod 97 = 13 (N)
			D(k, 23) &= 39 \cdot (23 - 13) \bmod 97 = 2 (C)
			D(k, 33) &= 39 \cdot (33 - 13) \bmod 97 = 4 (E)
		\end{align*}
		The encrypted message is (NCE).
	\end{enumerate}
	\item \textbf{Task 2}
	\item \textbf{Task 3}
	\item \textbf{Task 4}
	\item \textbf{Task 5}
\end{enumerate}
\end{document}