\documentclass{article}

\usepackage{homework}

\course{Encryption}
\assignment{Homework 2}

\begin{document}
	\homeworktitle
	
	\begin{enumerate}[label=\textbf{Task \arabic*:}]
		\item \textbf{Confusion and diffusion}
		\begin{enumerate}[label=\textit{Part \roman*:}]
			\item Encrypt \textsc{THEWANDCHOOSESTHEWIZARD} using Vigenére cipher with \textsc{MAGIC} as key: \textsc{fhkeczdipqaskavtecqbmrj}. I used my code from the previous assignment to achieve this. (In my code I used lowercase, meaning I had to convert the message and key in my code.)
			
			\item Change one letter in the plaintext: Encrypt \textsc{THEWANDCHOOSESTHELIZARD} using Vigenére cipher with \textsc{MAGIC} as key: \textsc{fhkeczdipqaskavterqbmrj}. I changed one letter compared to the original message, and only one letter changed in the resulting ciphertext.
			%Message: thewandchoosesthelizard encrypted with password: magic is:
			%fhkeczdipqaskavterqbmrj
			
			\item Change one letter in the key: Encrypt \textsc{THEWANDCHOOSESTHEWIZARD} using Vigenére cipher with \textsc{MANIC} as key: \textsc{fhreczdppqasravtejqbmrq}. I changed one letter in the key compared to the original key, and five letters changed in the resulting ciphertext (compared with the original ciphertext). 
			
			%Message: thewandchoosesthewizard encrypted with password: manic is:
			%fhreczdppqasravtejqbmrq
			%fhkeczdipqaskavtecqbmrj
			%00100001000010000100001			
		\end{enumerate}
		\item \textbf{Pseudorandom function}
		\item 
		\item 
		\item 
	\end{enumerate}
	
\end{document}