\documentclass{article}

\usepackage{homework}
\usepackage[breakable]{tcolorbox}

\addbibresource{references.bib}

\course{Encryption}
\assignment{Homework 3}

\begin{document}
	\homeworktitle
	
	\begin{enumerate}[label=\textbf{Task \arabic*:}]
		\item \textbf{RSA cryptosystem}\\
		The paper claims that with having $\phi(n)$ it is trivial to break RSA. But it does not tell anything about how one might gain $\phi(n)$, which is not easy for large numbers, that are generally used with RSA. Finding the Totient function ($\phi(n)$) generally requires to calculate all the primes that are smaller than $n$, which is a time consuming task.\\
		My advice based on this paper alone would be to keep using the already existing encryption system.
		\item \textbf{Security definitions}\\
		%What is the IND-OT-CPA? The adversary knows the public key(pk, n), sends a pair of messages m1 and m2, and gets one ciphertext back. Then he has to get which message was encrypted.
		Adversary can send $n$ as one message ($m_1$), whereas the other message ($m_2$) should not be divisible by $n$. This way, if the ciphertext is $0$, they can know, that $m_1$ was encrypted. This is due to the fact that remainders do not change with exponentiation.
		\item \textbf{ElGamal encryption}
			\begin{align*}
				Enc(pk, m): &\\
				y \leftarrow & {1, \ldots , q-1}\\
				s = & h^y \in \mathbb{G}\\
				E(pk, m) = & (c_1, c_2) = (g^y, m \cdot s) \in \mathbb{G} \times \mathbb{G}
			\end{align*}
		\begin{enumerate}[label=\textit{Part \roman*}]
			\item Create a re-randomized version of ElGamal:
			\begin{align*}
				y' \leftarrow & {1, \ldots , q-1} \quad \text{We generate a new randomness}\\
				s' = & h^{y'} \in \mathbb{G}\\
				\intertext{Then using the new randomness, we re-randomize the original message:}\\
				(c_{1}\prime, c_{2}\prime) = & (g^y \cdot g^{y'}, m \cdot h^y \cdot h^{y'}) = (g^{y+y'}, m \cdot h^{y+y'})
			\end{align*}
			Note that the new randomness is added on to the original one. 
			\item We can use the method previously described to create a new $c'$ with different randomness, that is valid under the same ElGamal scheme. Then we can give this new $c'$ to decrypt, and we will get back the original message, as it was not altered (only the randomness was changed).
			\item With $c_{11} = c_{12}$ we can deduce that the same randomness ($y$) was chosen, because $c_11 = g^y$, where $g$ is fixed for the scheme. Thus we can write the ciphertexts quotient as:
			\[ \frac{c_{21}}{c_{22}} = \frac{m_1 \cdot h^y}{m_2 \cdot h^y} = \frac{m_1}{m2} \]
			This way we know the ratio of the two messages, but we can't know their exact value.
			\item ElGamal's homomorphic property:\\
			\textit{Useful:} In electronic voting schemes. To tally the votes one can multiply the messages together to get a count of the votes. (For this application one needs to choose primes as message values.) \\
			\textit{Violates the security:} Messaging over insecure channel. An outsider can capture and forge new (potentially meaningful) messages without decryption.
		\end{enumerate}
		\item \textbf{Attack on RSA encryption}\\
		We have two captured ciphertexts of the same message:
		\begin{align*}
			m & & m &\\
			c_1 &= 7 & c_2 &= 16\\
			pk_1 &= (3, 57) & pk_2 &= (5,57)
			\intertext{Writing up the messages:}
			7 &= m^3 \mod 57 & 16 &= m^5 \mod 57
		\end{align*}
		We can notice that the base for both systems are the same. Thus we can use the Chinese Remainder Theorem to make breaking the message, using the fact that $57 = 3 \cdot 19$, where both factors are prime. First find $m$ in both remainders:
		\begin{enumerate}
			\item[Mod 3]
			\begin{align*}
				7 &= m^3 \mod 3 &\rightarrow 1 &= m^3 \mod 3\\
				16 &= m^5 \mod 3  &\rightarrow 1 &= m^3 \mod 3
			\end{align*}
			Then check the potential values:
			\begin{table}[h]
				\centering
				\begin{tabular}{r|r|r}
					\textbf{$m$} & \textbf{$m^3$} & \textbf{$m^5$} \\ \hline
					0 & 0 & 0 \\ 
					1 & 1 & 1 \\ 
					2 & 2 & 2 \\ 
				\end{tabular}
			\end{table}
			Thus we get $m=1 \mod 3$.
			\item[Mod 19]
			\begin{align*}
				7 &= m^3 \mod 19 &\rightarrow 7 &= m^3 \mod 19\\
				16 &= m^5 \mod 19  &\rightarrow 16 &= m^3 \mod 19
			\end{align*}
			Then check the potential values:
			\begin{table}[h]
				\centering
				\begin{tabular}{r|r|r}
					\textbf{$m$} & \textbf{$m^3$} & \textbf{$m^5$} \\ \hline
					0 & 0 & 0 \\ 
					1 & 1 &  \\ 
					2 & 8 &  \\ 
					3 & 8 &  \\ 
					4 & \emph{7} & 17 \\ 
					5 & 11 &  \\ 
					6 & \emph{7} & 5 \\ 
					7 & 1 &  \\ 
					8 & 18 &  \\ 
					9 & \emph{7} & 16 \\ 
					10 & 12 &  \\ 
					11 & 1 &  \\ 
					12 & 18 &  \\ 
					12 & 12 &  \\ 
					14 & 8 &  \\ 
					15 & 12 &  \\ 
					16 & 11 &  \\ 
					17 & 11 &  \\ 
					18 & 18 &  \\ 
				\end{tabular}
			\end{table} \\
			I only calculated those that are of interest for the $m^5$ column. We can see, that the result is $m = 9 \mod 19$
		\end{enumerate}
		Combining the results we get:
		\begin{gather*}
			m = 1 \mod 3 \\
			m = 9 \mod 19 \rightarrow m = 19k + 9 \\
			\text{Rewriting it to $\mod 3$:} \\
			19k + 9 = 1 \mod 3 \rightarrow 1k + 0 = 1 \mod 3 \\
			k = 1 \mod 3 \\
			k = 3t + 1 \\
			\text{Writing the result back to $m$:} \\
			m = 19k + 9 = 19(3t + 1) + 9 = 57t + 28
		\end{gather*}
		Thus we arrive to $m = 28$. (We have $\mod 57$, so the $t$ part can be ignored.)
		\item \textbf{Choosing cryptographic primitives}\\
		For this task, I used Claude's Sonnet model.\cite{claude3} This is the result:\\
		\begin{tcolorbox}[ 
			breakable,
			before skip=0pt,           % Remove extra space before the box
			after skip=12pt,            % Remove extra space after the box
			break at=\textheight       % Break the box at appropriate height]
			]
		\noindent\textbf{\Large Cryptographic Recommendations for System Requirements}
		
		\bigskip
		\noindent\textbf{\large Data Confidentiality and Integrity in Transmission}
		
		\noindent\textbf{Recommended schemes:}
		\begin{itemize}
			\item \textbf{Confidentiality:} AES-GCM (Advanced Encryption Standard with Galois/Counter Mode)
			\item \textbf{Integrity:} Inherent in GCM as an authenticated encryption mode
		\end{itemize}
		
		\noindent\textbf{Key length recommendation:} 256 bits
		
		\noindent\textbf{Justification:}
		\begin{itemize}
			\item AES-GCM provides both confidentiality and integrity in a single operation, making it efficient
			\item AES-256 provides a security margin well beyond current cryptanalytic capabilities
			\item GCM provides authentication tags to verify integrity while also encrypting the data
			\item For TLS implementation, TLS 1.3 with AES-256-GCM would be ideal
			\item Alternative: ChaCha20-Poly1305 for environments where AES hardware acceleration is unavailable
		\end{itemize}
		
		\bigskip
		\noindent\textbf{\large Data Confidentiality and Integrity at Rest}
		
		\noindent\textbf{Recommended schemes:}
		\begin{itemize}
			\item \textbf{Confidentiality:} AES-XTS for storage encryption
			\item \textbf{Integrity:} HMAC-SHA-256 stored alongside encrypted data
		\end{itemize}
		
		\noindent\textbf{Key length recommendations:}
		\begin{itemize}
			\item AES-XTS: 256 bits (effectively using two 256-bit keys)
			\item HMAC: 256 bits
		\end{itemize}
		
		\noindent\textbf{Justification:}
		\begin{itemize}
			\item AES-XTS is specifically designed for disk encryption and protects against ciphertext manipulation
			\item The XTS mode handles sector-based encryption properly, preventing certain attacks
			\item HMAC-SHA-256 provides strong integrity verification for stored data
			\item These key lengths provide protection against both classical and quantum attacks (AES-256 offers roughly 128-bit security against quantum attacks)
		\end{itemize}
		
		\bigskip
		\noindent\textbf{\large Strong Authentication Mechanisms}
		
		\noindent\textbf{Recommended schemes:}
		\begin{itemize}
			\item \textbf{User authentication:} Argon2id for password hashing
			\item \textbf{Service authentication:} TLS mutual authentication with X.509 certificates
			\item \textbf{API authentication:} HMAC-based tokens or JWT with EdDSA signatures
		\end{itemize}
		
		\noindent\textbf{Key length/parameter recommendations:}
		\begin{itemize}
			\item Argon2id: Memory cost $\geq$ 64MB, iterations $\geq$ 3, parallelism factor based on hardware
			\item TLS certificates: 
			\begin{itemize}
				\item RSA: 3072 bits
				\item ECDSA: 256 bits (curve P-256)
				\item EdDSA: Ed25519 (256 bits)
			\end{itemize}
			\item JWT: EdDSA with Ed25519 (256 bits)
		\end{itemize}
		
		\noindent\textbf{Justification:}
		\begin{itemize}
			\item Argon2id is the winner of the Password Hashing Competition, resistant to both timing and side-channel attacks
			\item TLS mutual authentication ensures both server and client verify each other's identities
			\item EdDSA provides strong performance advantages while maintaining security
			\item These key sizes provide approximately 128 bits of security, the current standard for long-term protection
		\end{itemize}
		
		\bigskip
		\noindent\textbf{\large Resistance to Common Attacks}
		
		\noindent\textbf{Side-channel attack resistance:}
		\begin{itemize}
			\item Use constant-time implementations of cryptographic algorithms
			\item Employ hardware security modules (HSMs) for key operations
			\item Implement memory zeroization after key usage
		\end{itemize}
		
		\noindent\textbf{Brute force attack resistance:}
		\begin{itemize}
			\item For key generation: Use cryptographically secure random number generators (CSPRNG)
			\item For password hashing: Use Argon2id with appropriate memory and time cost parameters
			\item For session tokens: High-entropy random generation ($\geq$ 256 bits)
		\end{itemize}
		
		\noindent\textbf{Key management recommendations:}
		\begin{itemize}
			\item Key rotation policies (e.g., TLS certificates every 1 year, symmetric keys every 90 days)
			\item Secure key storage using HSMs or secure enclaves
			\item Implement proper key derivation functions (HKDF) when deriving multiple keys
		\end{itemize}
		
		\bigskip
		\noindent\textbf{\large Post-Quantum Considerations}
		
		\noindent Given the requirements for resistance to future attacks, I recommend planning for post-quantum security:
		
		\medskip
		\noindent\textbf{1. For symmetric encryption:}
		\begin{itemize}
			\item Continue using AES-256, which remains quantum-resistant
		\end{itemize}
		
		\noindent\textbf{2. For asymmetric cryptography:}
		\begin{itemize}
			\item Begin transition planning to quantum-resistant algorithms:
			\begin{itemize}
				\item Consider CRYSTALS-Kyber for key encapsulation (replacing RSA/ECDH)
				\item Consider CRYSTALS-Dilithium for digital signatures (replacing RSA/ECDSA)
			\end{itemize}
			\item Implementation via hybrid approaches during transition
		\end{itemize}
		
		\noindent\textbf{3. For hash functions:}
		\begin{itemize}
			\item SHA-256/SHA-3 remain secure against known quantum attacks
		\end{itemize}
		
		\medskip
		\noindent This approach ensures the system meets current security requirements while providing a pathway to quantum resistance as these algorithms continue to mature and become standardized.
		\end{tcolorbox}
		
		\textit{My evaluation:}\\
		Good.
	\end{enumerate}
	
	\printbibliography
\end{document}